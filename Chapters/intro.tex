\chapter*{Introduction}\label{intro}
\lhead{\emph{Introduction}}

Différents critères peuvent nous servir à séparer de façon naturelle ce qui est vivant de ce qui ne l'est pas:
\begin{description}
\item[$\bullet$] \textbf{La structure:} les organismes vivants sont organisés en une ou plusieurs cellules.
\item[$\bullet$] \textbf{L'information:} Tout organisme vivant présente, au sein de ses cellules, de l'information qui est codée au niveau de l'ADN.
\item[$\bullet$] \textbf{La continuité:} Au cours de sa vie, un organisme vivant doit mettre en place des mécanismes assurant la transmission conforme de son matériel génétique à sa descendance permettant ainsi sa viabilité.
\end{description}

Les architectures des génomes bactériens sont de même soumises à ces principes et il a été caractérisé, au sein de divers génomes bactériens, de nombreux mécanismes génétiques permettant la transmission de l'information génomique (Chapitre \ref{chap1a}). Sur cette base, le génome est organisé en \textbf{réplicon(s)}, ou unité(s) de réplication, (\S \ref{architecture}) qui, chez les bactéries,  correspondent à des chromosomes ou des plasmides selon leur caractère essentiel ou accessoire pour l'organisme hôte (\S \ref{essacc}). En plus de leur rôle de support de l'information génétique, les réplicons bactériens sont des éléments fluides où prennent place différents phénomènes de transfert intra- ou inter-génomique, participant à l'adaptation et l'évolution du matériel génétique et des organismes biologiques. \\
L'ensemble des connaissances des \textbf{S}ystèmes de \textbf{T}ransmission de l'\textbf{I}nformation \textbf{G}énétique (STIG) provient de l'étude d'un petit nombre d'espèces bactériennes et semblent montrer l'existence de caractéristiques spécifiques des STIG des chromosomes, nécessitant une stabilisation dans le cycle cellulaire, et des STIG des plasmides, pouvant permettre un comportement pseudo-autonome de ces derniers (\S \ref{parevoldesrepl}). La conception traditionnelle du génome bactérien est qu'il est constitué d'un unique chromosome présentant des mécanismes d'intégration dans le cycle cellulaire découplés de ceux des autres réplicons additionnels présents dans la cellule. La découverte assez récente de \textbf{R}éplicons \textbf{E}xtra-\textbf{C}hromosomiques \textbf{E}ssentiels (RECE), réplicons structurellement différents du chromosome bien que possédant des gènes essentiels (Chapitre \ref{chap1b}), pose alors la question des mécanismes génétiques d'intégration des RECE dans le cycle cellulaire en parallèle des chromosomes (\S\ref{model}). Au delà de l'aspect mécanistique, le rôle et l'origine des RECE sont des problématiques ouvertes dont la compréhension peut apporter des éléments de réponse quant à la complexification du génome et des organismes vivants par le passage entre structure génomique monopartite, à un chromosome, et  structure multipartite, à plusieurs chromosomes (\S\ref{justification}).\\
On peut faire l'hypothèse que les RECE, en tant qu'espèce génomique propre, sont dans un état d'intégration du cycle cellulaire distinct des chromosomes et des plasmides. Les STIG, mécanismes génétiques responsable de cette intégration, devraient alors pouvoir fournir des critères de discrimination des plasmides, chromosomes et RECE (Chapitre \ref{chap2}). L'approche alors suivie est de caractériser les réplicons bactériens par leurs STIG. 
\\
\\
Plus formellement, on peut modéliser l'hypothèse de la façon suivante:
$$
Stabilisation = f(STIG)
$$
L'objectif est alors d'inférer le modèle $f$ décrivant la réalité et inconnu. Une première remarque est que l'on peut naturellement supposer que le modèle réel $f$ prend en compte des paramètres $\lambda$ additionnels et inconnus: 

$$Stabilisation = f(STIG,\lambda)$$

Par exemple, on peut supposer que l'état de stabilisation d'un réplicon est fortement couplé à l'écologie de l’hôte bactérien (comme ce qui semble exister chez les Rhizobiales). Cependant il est raisonnable de penser que les STIG eux-mêmes reflètent ces paramètres externes : un réplicon additionnel fortement stabilisé d'un hôte symbiote aura des STIG spécifiques. Ainsi, l'utilisation des STIG est suffisante pour modéliser $f$.\\
\\
Le travail présenté dans cette thèse a consisté à proposer des modèles $\hat{f}$ approximés de $f$, déduits d'ensembles de données $\widehat{STIG}$ liés aux STIG et $\widehat{Stabilisation}$ liés à la stabilisation des génomes. On a ainsi la relation suivante:
$$
\widehat{Stabilisation} = \hat{f}(\widehat{STIG}) + erreur
$$
où l'erreur est d'autant plus faible que $\hat{f}$ est probablement proche de $f$.  

Les jeux de données $\widehat{Stabilisation}$ sont construits à partir d'un ensemble de génomes et réplicons et de leurs annotations (chromosome, plasmide ou RECE) par les bases de données publiques dont ils sont extraits (Chapitre \ref{chap3b}). Les données $\widehat{STIG}$ utilisées sont construites à partir de clusters de protéines homologues en terme de séquence et de domaines fonctionnels, qui sont sélectionnés initialement par leurs homologies de séquence envers des protéines annotées fonctionnellement et en lien avec les STIG (Chapitre \ref{chap3b}). Les clusters de protéines sont ensuite employés comme attributs des réplicons bactériens pour \textit{structurer} l'espace des réplicons par des analyses non-supervisées de clustering et de visualisation (Chapitre \ref{chap4b}). Les clusters de protéines sont obtenus en utilisant un ensemble d'une centaine de fonctions liées aux STIG (Annexes \ref{AppendiceB} et \ref{AppendiceC}). Ces fonctions sont ensuite directement utilisées pour caractériser les différents réplicons et génomes (mono- et multi-partites) bactériens (Chapitre \ref{chap6true}). Ces premières études permettent de mettre en évidence des spécificités des différents types de réplicons, ce qui permet, avec l'utilisation d'algorithmes d'apprentissage supervisé, d'identifier de nouveaux RECE potentiels parmi les réplicons bactériens (Chapitre \ref{chap6}). Enfin, afin de mieux comprendre les mécanismes évolutifs impliqués dans la formation des génomes multipartites, des analyses complémentaires de synténie sont réalisées entre génomes multi- et monopartites (Chapitre \ref{chapsynt}). Les résultats de ces différentes analyses permettent ainsi de qualifier les différences fonctionnelles en terme de STIG entre les réplicons bactériens, et mènent à proposer des modèles évolutifs de formation des génomes multipartites et à souligner la continuité du matériel génomique bactérien (Chapitre \ref{discussionG}).
